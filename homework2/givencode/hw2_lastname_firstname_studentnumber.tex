\documentclass[a4paper]{article}
\usepackage[a4paper, margin=3cm]{geometry}
\usepackage{framed}
\usepackage{listings}
\usepackage{graphicx}
\usepackage{epstopdf}
\usepackage{xcolor}
\usepackage{amsmath}
\usepackage{pgfplots}

\title{Homework Assigment 2: Blocking and cache-efficiency\\ \large Scientific Software / Technisch Wetenschappelijke Software}

\newcommand{\answer}[1]{\vspace{-0.75em}\begin{framed} #1 \end{framed}\vspace{-0.75em}}

\definecolor{ferngreen}{HTML}{56641a}
\definecolor{perfumepurple}{HTML}{c0affb}
\definecolor{apricotorange}{HTML}{e6a176}
\definecolor{orientblue}{HTML}{00678a}
\definecolor{winered}{HTML}{984464}
\definecolor{downygreen}{HTML}{5eccab}

\author{Firsname Lastname, replace with your name!!!} % Update with your name
\date{}
\begin{document}

\maketitle

\IfFileExists{./time_spent.txt}{}{\textcolor{red}{\textbf{Could not find a file named \texttt{time\_spent.txt} in this directory! Make sure to include such a file containing the number of hours you spent on this assignment as part of your submission! Then recompile this report in order to remove this warning.}}}

\section*{Practical info}
Machine name: \textbf{ Do not forget to update this.} % Update to match the machine you used to perform timings

\paragraph{Other sources} Did you use any sources other than the course material? Cite them here. Did you use generative AI? If so, how did you use it?
\answer{}

\subsection*{Questions}
% NOTE: you can use mmm to abbreviate matrix-matrix multiplication
\begin{itemize}
	\item[\textbf{Q1}:] Compile \texttt{time\_matmul.cpp} with the \texttt{g++} compiler and the following compiler flag combinations\footnote{More information on compiler options that control the optimization level for the \texttt{g++} compiler can be found on \texttt{https://gcc.gnu.org/onlinedocs/gcc/Optimize-Options.html}. }: 
    \begin{itemize}
        \item \texttt{-O0}
        \item \texttt{-Og}
        \item \texttt{-O3}
        \item \texttt{-O3 -DNDEBUG}
        \item \texttt{-O3 -DNDEBUG -march=native}
    \end{itemize}
    Compare the results of \texttt{matmul\_naive} for $n=1200$. What is the role of the different compiler flags? What do you conclude? When would you use which combination? Which combination will you use for the rest of the report? (max. 10 lines)
	\answer{}

	\item[\textbf{Q2}:] Describe what techniques you used in \texttt{time\_matmul.cpp} to obtain accurate timings. (max. 15 lines)
	\answer{}
	
	\item[\textbf{Q3}:] Because it does $33\%$ fewer operations, you would expect \texttt{matmul\_naive\_v2} to be $33\%$ faster than \texttt{matmul\_naive}. Do you actually see this speedup? Explain. (max. 5 lines)
	\answer{}

	\item[\textbf{F1}:] Make a figure that shows the number of floating point operations per second in function of $N$ for the \texttt{matmul\_naive\_v2} and \texttt{matmul\_reordered}.
	\begin{figure}[!h]
		\centering
		\begin{tikzpicture}
			\begin{semilogyaxis}[
				xlabel=$n$,
				xtick={24,480,960,1440,1920,2400},
				ylabel=GFLOPS,
				ytick={0.25,0.5,1,2,4,8,16},
				yticklabels={0.25,0.5,1,2,4,8,16},
				legend style={at={(0.5,-0.2)},anchor=north},
				width=0.5\textwidth,
				grid=both,
                mark size=1.5pt,
                cycle list={
                    {ferngreen,mark=square},
                    {perfumepurple,mark=o},
                    {apricotorange,mark=+},
                    {orientblue,mark=triangle},
                    {winered,mark=otimes},
                    {downygreen,mark=*}
                }, 
			]

			\addplot table [x=n, y=gflops] {plotdata.txt};
			\addlegendentry{naive}

			\addplot table [x=n, y=gflops] {plotdata2.txt};
			\addlegendentry{reordered}

			\end{semilogyaxis}
		\end{tikzpicture}
	\end{figure}

	\item[\textbf{Q4}:] Explain the difference in execution time between \texttt{matmul\_naive\_v2} and \texttt{matmul\_reordered}. Try to be as detailed as necessary, but remain to the point. Highlight the most important concepts using \textbf{boldface}. Make sure to refer to the figure. (max. 20 lines)
	\answer{}

	\item[\textbf{Q5}:] What is blocking? Give some advantages and disadvantages of the technique. (max. 5 lines)
	\answer{}
	
	\item[\textbf{Q6}:] You can use \texttt{valgrind --tool=cachegrind --cache-sim=yes ./executable} with \texttt{executable} the name of your executable to simulate the cache architecture of your machine\footnote{To inspect the size of the caches of your machine, you can run \texttt{getconf -a | grep CACHE}.} for the given executable. For the naive and reordered implementation, compare the cache efficiency of the variant without blocking and the blocked variant for blocksizes 25, 100 and 500. Also compare the execution time of these variants (run them without \texttt{valgrind}). Don't use \texttt{time\_matmul.cpp} for this question but write a separate short program. Use $N$ equal to 2000.\footnote{Cachegrind simulates the memory and significantly increases the runtime. Don't be surprized if a single run takes 5 to 10 minutes. Try to make sure your code works before attempting this question, otherwise you will have to do it again.} Do you get the results you expect based on the previous questions? Discuss. (max. 8 lines)
	\begin{table}[!h]
		\centering
		\begin{tabular}{ll|ccc}
			& & L1 cache misses & LL cache misses & Execution time \\ \hline
			Naive & No blocking & & & \\
			& Blocksize = 25 & & & \\
			& Blocksize = 100 & & & \\
			& Blocksize = 500 & & &  \\[10pt] \hline 
			Reordered & No blocking & & & \\
			& Blocksize = 25 & & & \\
			& Blocksize = 100 & & & \\
			& Blocksize = 500 & & & 
		\end{tabular}
	\end{table}
	\answer{}
	
	\item[\textbf{F2}:] As in F1, plot the number of floating point operations per second in function of $N$ for \texttt{matmul\_reordered}, \texttt{matmul\_blocks} and \texttt{matmul\_blocks\_b}
	\begin{figure}[!h]
		\centering
		\begin{tikzpicture}
			\begin{semilogyaxis}[
				xlabel=$n$,
				xtick={24,480,960,1440,1920,2400},
				ylabel=GFLOPS,
				ytick={0.25,0.5,1,2,4,8,16},
				yticklabels={0.25,0.5,1,2,4,8,16},
				legend style={at={(0.5,-0.2)},anchor=north},
				width=0.5\textwidth,
				grid=both,
                mark size=1.5pt,
                cycle list={
                    {ferngreen,mark=square},
                    {perfumepurple,mark=o},
                    {apricotorange,mark=+},
                    {orientblue,mark=triangle},
                    {winered,mark=otimes},
                    {downygreen,mark=*}
                }, 
			]

			\addplot table [x=n, y=gflops] {plotdata.txt};
			\addlegendentry{reordered}

			\addplot table [x=n, y=gflops] {plotdata.txt};
			\addlegendentry{Blocks}

			\addplot table [x=n, y=gflops] {plotdata.txt};
			\addlegendentry{Blocks b}

			\end{semilogyaxis}
		\end{tikzpicture}
	\end{figure}
	
	\item[\textbf{Q7}:] Explain the difference in execution time between the variants plotted in F2. Try to be as detailed as necessary, but remain to the point. Highlight the most important concepts using \textbf{boldface}. Make sure to refer to the figure. (max. 5 lines)
	\answer{}
	
	\item[\textbf{F3}:] As in F1 and F2, plot the number of floating point operations per second in function of $N$ for \texttt{matmul\_reordered}, \texttt{matmul\_blocks}, \texttt{matmul\_recursive}, and \texttt{matmul\_kernel}. Also include a line that indicates the peak performance of your machine.
	\begin{figure}[!h]
		\centering
		\begin{tikzpicture}
			\begin{semilogyaxis}[
				xlabel=$n$,
				xtick={24,480,960,1440,1920,2400},
				ylabel=GFLOPS,
				ytick={0.25,0.5,1,2,4,8,16},
				yticklabels={0.25,0.5,1,2,4,8,16},
				legend style={at={(0.5,-0.2)},anchor=north},
				width=0.5\textwidth,
				grid=both,
                mark size=1.5pt,
                cycle list={
					{black, mark=none},
                    {ferngreen,mark=square},
                    {perfumepurple,mark=o},
                    {apricotorange,mark=+},
                    {orientblue,mark=triangle},
                    {winered,mark=otimes},
                    {downygreen,mark=*}
                }, 
			]

			% A quick way to add a horizontal line is to override the y data for
			% one of the plots. This is a bit of a hack, but it works.
            \addplot table [x=n, y expr={10.0}] {plotdata.txt};
            \addlegendentry{Peak performance}

			\addplot table [x=n, y=gflops] {plotdata.txt};
			\addlegendentry{reordered}

			\addplot table [x=n, y=gflops] {plotdata.txt};
			\addlegendentry{Blocks}

			\addplot table [x=n, y=gflops] {plotdata.txt};
			\addlegendentry{Recursive}

			\addplot table [x=n, y=gflops] {plotdata.txt};
			\addlegendentry{Kernel}

			\end{semilogyaxis}
		\end{tikzpicture}
	\end{figure}
	
	\item[\textbf{Q8}:] Briefly explain how you calculated the peak performance. (max. 5 lines)
	\answer{}
	
	\item[\textbf{Q9}:] Discuss the recursive matrix-matrix multiplication. What are the advantages and disadvantages of this method? Also discuss the threshold when you switch to a simple method with loops. Was it easy to select a good threshold? (max. 7 lines)
	\answer{}
	
	\item[\textbf{Q10}:] Discuss the matrix-matrix multiplication that uses the kernel. (max. 5 lines)
	\answer{}


\end{itemize}
\end{document}